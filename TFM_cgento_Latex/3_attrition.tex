\chapter{Panel attrition: causas, presencia en la EFF y predicción}
\label{chapter:attrition}
\section{Causas del panel attrition y cómo reducir su impacto durante la recolección de datos}

Conceptualmente, las causas del panel attrition pueden clasificarse en tres categorías condicionales (\cite{lepkowski2002nonresponse}): no-localización, no-contacto y no-cooperación.

La no-localización se refiere a no localizar exitosamente a un encuestado durante una ola posterior. Generalmente, esto se debe a cambios en la información de contacto (dirección de residencia, número teléfono, correo electrónico...) obtenida del participante durante la ola anterior (\cite{couper2009keeping}). Algunos factores que pueden contribuir al éxito o fracaso en la localización son el método de recolección de datos, la propensión a los cambios de localización de los encuestados entre diferentes olas, el tiempo transcurrido entre olas e incluso el presupuesto (\cite{lynn2009methods}). Por ejemplo, las encuestas con entrevistas cara a cara utilizan métodos de rastreo y búsqueda que suelen ofrecer altos índices de localización y cooperación (\cite{de2005mix}, \cite{couper2009keeping}), pero también requieren esfuerzos adicionales para localizar a los participantes panel que se hayan mudado, como por ejemplo reembolsar a los entrevistadores los gastos derivados del proceso de búsqueda.

Posteriormente, tras localizar al encuestado, es necesario establecer un contacto. En este paso el método de recolección de datos vuelve a ser muy importante. Por ejemplo, en una encuesta por correo postal o por email, el contacto depende de que el encuestado reciba dicho correo y además le preste atención. En cambio, en entrevistas presenciales en el hogar o entrevistas telefónicas, debe darse una coincidencia temporal entre la disponibilidad del entrevistado (está en casa o tiene su teléfono disponible) y el intento de contacto del entrevistador (realiza una visita su casa o llama por teléfono). En este sentido, dos estrategias de contacto que han presentado buenos resultados han sido utilizar un número de intentos de contacto alto y diversificado en horarios (mañana, tarde, fines de semana...), y establecer períodos para realizar entrevistas lo suficientemente largos (\cite{nicoletti2005survey}, \cite{watson2009identifying}). Además, las encuestas longitudinales ofrecen la ventaja de poder utilizar datos sobre el proceso de recolección en olas anteriores, y diseñar mejores estrategias de contacto o identificar casos en los que el contacto puede ser complicado (\cite{calderwood2012using}, \cite{lagorio2016call}).

Finalmente, cuando se ha establecido contacto con el encuestado. éste puede cooperar de nuevo con los encuestadores, o por el contrario rechazar hacerlo. La falta de cooperación es una preocupación común en las encuestas, y hay bastante literatura sobre las motivaciones que puede haber detrás de los rechazos. Suelen destacar factores relacionados con las características de los participantes, como características sociodemográficas (edad, sexo, nivel de salud...) o el entorno (composición del hogar, del barrio de resdiencia...); con el diseño de la encuesta, como la temática de las preguntas o el método de recolección de datos; y con las características de los entrevistadores, como su apariencia o la experiencia previa realizando encuestas (\cite{groves1992understanding},\cite{obrien2006sensitive}. En el caso de las encuestas longitudinales, los encuestados ya tienen una experiencia previa sobre la encuesta que puede hacer potenciar el efecto de factores como la duración de la entrevista, la carga cognitiva que supone pensar las respuestas, la sensibilidad sobre la temática o la fatiga por haber participado ya en varias ediciones (\cite{laurie1999strategies}, \cite{watson2009identifying}, \cite{lynn2018tackling}). Pero, al igual que con los contactos, las encuestas longitudinales también poseen información sobre el proceso de contacto, el desarrollo de la entrevista y las características de los encuestados en olas anteriores que pueden ayudar a diseñar incentivos que contribuyan a la cooperación (\cite{laurie2009use}) o descubrir la influencia de los entrevistadores para convencer a los encuestados, y lo importante que es su continuidad entre olas (\cite{lynn2014continuity}).

\section{Falta de respuesta y panel attrition en la Encuesta Financiera de las Familias}

La Encuesta Financiera de las Familias es una encuesta longitudinal en la que se entrevista a hogares residentes en España. Su primera edición se realizó en el año 2002, y se ha producido de manera trienal hasta el año 2020. Desde entonces, su producción es bienal, siendo la ola de 2020 la primera planificada para ser publicada con esta nueva frecuencia.

El objetivo de su primera edición era captar bien la distribución de la riqueza de los hogares españoles, por lo que el diseño de su muestra se fundamentó en un sobremuestreo de hogares con mayor nivel de riqueza\footnote{El diseño del sobremuestreo de la EFF toma como referencia la \href{https://www.federalreserve.gov/econres/scfindex.htm}{Survey of Consumer Finances (SCF)}, que es la encuesta equivalente a la EFF en Estados Unidos, y que elabora la Reserva Federal. Lo ejecuta el Instituto Nacional de Estadística, con la colaboración de la Agencia Tributaria. El nivel de riqueza de los hogares se obtiene a partir de la declaraciones individuales más recientes en el impuesto sobre el patrimonio que hay en España (facilitado por la Agencia tributaria), y se definen unos estratos de riqueza a partir de los intervalos de la SCF y los percentiles de la distribución del nivel de riqueza de los hogares. Los hogares en los estratos más altos tienen mayor probabilidad de ser seleccionados. Para las regiones de País Vasco y Navarra no se realiza sobremuestreo porque no se dispone de información} (\cite{effmethod2002}). Desde la siguiente edición (EFF2005) en adelante, además de mantener la representatividad de la muestra para el año de la ola correspondiente, se incluyó como objetivo adicional el incluir un componente longitudinal que consistía en volver a preguntar a todos los hogares que participaron en la ola anterior, y así poder hacer análisis de causalidad. Para conseguir ambos objetivos, en las ediciones de 2005, 2008 y 2011 se implementaron muestras de refresco por estrato de riqueza para complementar a la muestra panel que venía de olas anteriores (\cite{effmethod2005}, \cite{effmethod2008},  \cite{effmethod2011}). Finalmente, en la edición de la EFF2014 se estableció el límite máximo de participación de los hogares en cuatro olas consecutivas, y desde entonces se eliminan de la muestra longitudinal a aquellos hogares que ya han participado en cuatro olas consecutivas (\cite{effmethod2014}, \cite{effmethod2017}).

El proceso de recopilación de los datos se basa en la realización de entrevistas personales, generalmente en la residencia de los hogares seleccionados. Los entrevistadores deben localizar la vivienda en la que reside del hogar, realizar varias visitas en diferentes combinaciones de horarios y días de la semana\footnote{De manera general, unos días antes de la primera visita, la empresa de campo envía una carta al hogar, firmada por el gobernador del Banco de España, para informarle de que ha sido seleccionado para participar en la EFF, y que en los próximos días recibirá la visita de una persona para concertar una cita para una entrevista personal. A algunos hogares panel también se les contacta por teléfono si mostraron alguna preferencia por ser contactados de esa manera.}, pedir su colaboración con la EFF y concertar una cita para realizar la entrevista\footnote{Los hogares que aceptan participar y completan la entrevista reciben un obsequio por parte del Banco de España. Los hogares panel también reciben otro obsequio por haber participado en la edición anterior, independientemente de que accedan a participar de nuevo.}. Este procedimiento es el mismo tanto para la muestra de refresco como para la muestra panel. La participación es completamente voluntaria, la información es confidencial, y los hogares pueden decidir interrumpir la entrevista o pararla en cualquier momento.

El Banco de España incluye un análisis de la falta de respuesta y de panel attrition en los documentos metodológicos de cada ola de la EFF (\cite{effmethod2002}, \cite{effmethod2005}, \cite{effmethod2008}, \cite{effmethod2011}, \cite{effmethod2014}, \cite{effmethod2017}). En estos análisis se muestran, diferenciando entre muestra panel y muestra no panel, el número de hogares contactados por tipo de respuesta (completas, rechazos, no contactados...), las tasas de cooperación según el estrato de riqueza de los hogares contactados, y los odd-ratios de dos modelo logit de cooperación frente a rechazo en el que se utilizan diferentes categorías recogidas por los entrevistadores sobre las condiciones de los edificios, del nivel socioeconómico del barrio, el tamaño del municipio de residencia en número de habitantes y la región de residencia. En general se observa que se contacta con más hogares no panel que con hogares panel, pero es un resultado esperable porque sus tasas de cooperación también son más bajas que las de la muestra panel. Esto se mantiene tanto para la totalidad de la muestra como diferenciando por estrato de riqueza. También se observa que para ambos tipos de muestra las tasas de cooperación tienden a ser menores a medida que aumenta el estrato de riqueza de los hogares. Con respecto a los modelos logit, los resultados difieren entre entre olas ya que en una ola son estimadores son significativos, pero en otra no lo son. Aún así, algunas tendencias sí se han observado de manera repetida en la mayoría de olas son que la probabilidad de cooperar decrece cuando aumenta el tamaño del municipio\footnote{Este efecto siempre es negativo para la muestra no panel en todas las olas, pero para la muestra panel hay algunas olas para las que su efecto no es significativo.}, que la probabilidad de no cooperar aumenta para barrios con menor nivel económico, y que existen diferencias en la cooperación entre regiones. Sin embargo, es posible que el efecto regional esté afectado por algún tipo de efecto de entrevistador, ya que los entrevistadores suelen hacer sus entrevistas en las mismas regiones ola tras ola.

Estos análisis que acabamos de describir muestran que existe no respuesta y panel attrition en la EFF. Pero se limitan a hacer una comparación de cada edición con la inmediatamente anterior. No se observa cuántos hogares suelen completar las cuatro olas para las que han sido seleccionadas, ni la distribución de los abandonos con el paso de las olas. Por otro lado, a la hora de analizar la no respuesta de muestra panel no se utiliza información recogida durante la ola anterior que podría ser relevante para analizar el panel attrition, como por ejemplo las características de los hogares, el número de intentos de contacto en la edición anterior, o la duración de la entrevista de la ola anterior.

La EFF ofrece muchas oportunidades para poder investigar sobre la falta de respuesta y el panel attrition en esta encuesta, ya que hay muchos análisis que no se han podido realizar. Conocer las causas que pueden provocar el panel attrition en la EFF ayudará a desarrollar herramientas que ayuden a combatirlo, y con ello mejorar las tasas de respuesta de los hogares y la calidad de los resultados de la encuesta.

\section{Predicción de Panel Attrition}

La regla tradicional utilizada para diseñar encuestas ha sido la estandarización de todos los procesos y protocolos. Todas unidades muestrales deben ser tratadas de la misma manera. Con la excepción de las introducciones de los entrevistadores (\cite{groves1992understanding}), esta regla se mantuvo durante bastante tiempo. Pero el aumento de las reticencias de la población a participar en encuestas y los recortes presupuestarios llevó a buscar cómo mejorar la eficiencia de los procesos de elaboración de encuestas, y empezaron a considerarse diseños adaptativos enfocados a subgrupos específicos de la muestra (\cite{groves2006responsive}, \cite{lynn2014targeted}, \cite{lynn2017standardised}). En ese sentido, la predicción se ha convertido en una opción muy interesante para poder identificar anticipadamente a individuos que potencialmente podrían abandonar precipitadamente una encuesta longitudinal, y en los que los métodos de machine learning tienen un peso importante ya que no necesitan un conocimiento previo de las relaciones que se quieren estudiar, y suelen adaptarse bien a contextos en los que la relación entre la variable dependiente y sus predictores suele ser compleja y no lineal (\cite{buskirk2018introduction}, \cite{kern2019tree}, \cite{kern2021predicting}, \cite{jankowsky2022validation}).

Los estudios de modelos predictivos realizan una comparación de rendimiento entre modelos tradicionales utilizados para analizar panel attrition, casi siempre una regresión logística, y modelos basados en métodos de Machine Learning, como diferentes tipos de árboles de decisión o máquinas de soporte de vectores. \cite{kern2019tree} y \cite{kern2021predicting} muestran casos en los que los modelos predictivos basados en machine learning, especialmente árboles de decisión, presentan resultados prometedores. Sin embargo, en \cite{jankowsky2022validation} apenas observan diferencias significativas entre los resultados de un modelo de regresión logística y los de un modelo GBM (Gradient Boosting Machine). La conclusión de ese artículo es que no necesariamente un modelo más complejo (y más complejo de entender) puede ser más adecuado para predecir panel attrition, y por tanto utilizarse para diseñar políticas adaptativas para reducirlo.

La intención de este proyecto es realizar un estudio similar a los propuestos por \cite{kern2021predicting} y \cite{jankowsky2022validation}, y comprobar si modelos basados en métodos de machine learning pueden predecir adecuadamente el panel attrition en la EFF, y dar la posibilidad de contribuir al desarrollo de herramientas que ayuden a reducirlo.