\chapter{Conclusiones y futuras líneas de trabajo}
\label{chapter:conclusiones}

En este Trabajo de Final de Master se ha intentado aplicar la implementación de \cite{beste2023case} para predecir la no participación de hogares panel en encuestas longitudinales en olas futuras al caso de la Encuesta Financiera de las Familias. Para ello, se ha usado un modelo de Regresión Logística como referencia, y se han entrenado modelos CART, Random Forest, XGBooster y Naïve Bayes. El conjunto de predictores está formado por variables que potencialmente pueden explicar la participación de un hogar en la edición siguiente de la encuesta, como por ejemplo el interés mostrado por la persona que respondió a la encuesta, el nivel de recelo que mostró durante la entrevista, su nivel de salud o si consintió que la entrevista fuese grabada.

Los modelos de Random Forest y XGBooster presentaron mejores rendimientos que el modelo de referencia Logit en todas las métricas de evaluación consideradas, pero el valor de ROC AUC del mejor de los modelos no superó el valor de 0.6, lo cual lo clasifica como un predictor malo. Para intentar aprender sobre cómo se ha hecho la predicción, se hizo el ejercicio de usar los mismos modelos para hacer la predicción con el conjunto de entrenamiento. El modelo que funcionó mejor fue el Random Forest, pero su ROC AUC no superó el valor de 0,7, que lo clasifica como un predictor regular. Al observar las variables que más importancia tuvieron durante el entrenamiento del Random Forest destabacan que la persona que contestó a la estrevista a ya formase parte del hogar desde al menos dos ediciones antes, el valor del interés que mostró también era importante, y que el entrevistador considerase que el hogar participó por la relevancia de la encuesta.

A partir de los resultados que se han visto en este proyecto, se plantean las siguientes reflexiones y los posibles pasos que se podrían dar en el futuro para este proyecto:

\begin{enumerate}
    \item La exploración de los datos sugiere que las variables seleccionadas para entrenar los modelos de predicción guardan cierta relación con la variable de attrition. Pero no son suficientes para predecir bien si un hogar panel dejará de participar en la ola siguiente. Tal y como hemos comentado al principio de este capítulo, la EFF genera una gran cantidad de variables, y se hizo un filtrado inicial basado en las implementaciones de \cite{beste2023case} y \cite{kern2021predicting}. Es posible que haya \textbf{variables que no se han incluido, y que tengan poder para predecir el attrition}. Una vía de trabajo para el futuro es \textbf{volver a revisar toda la información disponible y plantear una nueva selección de variables}. El uso de métodos de machine learning para selección de variables es una opción que se podría implementar.
    \item Las olas de la EFF seleccionadas para el estudio son las de los años 2017, 2020 y 2022 porque son las que tienen mayor cantidad de datos y también los de mayor calidad. Todos los modelos entrenados buscan utilizar datos de lo que había en 2017 para predecir algo que iba a ocurrir en 2020, que es el año en el que tuvo lugar la crisis del \textbf{Covid-19}. Y posteriormente, en el test, se utiliza información recogida durante 2020 para predecir lo ocurrido en el año 2022, cuando el mundo estaba ya superando la crisis del Covid-19. El año 2020 fue un año especial en la EFF porque seguramente mucha gente fuera más reticente a participar por el covid, lo cual afectaría a la variable target del entrenamiento). También se tuvo que cambiar la metodología de la entrevista, pasando de entrevista personal a entrevista telefónica, que afecta a la calidad de los datos (\cite{lynn2018tackling}), y por tanto a los datos de los predictores usados en el test. Ante esta problemática, se plantean dos posibles alternativas:
    \begin{enumerate}
        \item Las olas de \textbf{EFF2002 a EFF2017 son homogéneas} en lo que a metodología se refiere. Todas las entrevistas fueron personales, y no hubo una crisis como la del Covid-19. Una alternativa interesante sería crear \textbf{nuevos modelos de predicción}, pero utilizando sólo la información disponible en esas olas. Serían \textbf{menos variables} (las respuestas de los hogares y la información recogida por los entrevistadores), pero podría ser que el rendimiento fuese mejor.
        \item La EFF va a continuar realizándose en los próximos años, y con una frecuencia bienal. Se puede volver a \textbf{plantear esta misma metodología dentro unos años, utilizando sólo ediciones completadas después de 2020}, y con el beneficio de recoger toda la información que se ha estado recogiendo en las últimas ediciones y que no está disponible para antes de 2017.
    \end{enumerate}
    \item Los resultados de las predicciones sobre los datos de entrenamiento también podría explicarse por la \textbf{existencia de información no observable en el momento de hacer la predicción, y que además tenga más peso para determinar el resultado de la participación que la información recabada durante la ola anterior}. El Covid-19 es un buen ejemplo de algo que no se puede prever, pero otra cosa que también se desconoce es qué entrevistadores habrá durante la siguiente edición. En todas las ediciones hay entrevistadores nuevos, y algunos funcionan muy bien, y otros no. Tal y como comentan \cite{lynn2018tackling} y \cite{groves2006nonresponse}, el papel del entrevistador es importante para la colaboración de los hogares y la calidad de los datos. Esto puede investigarse identificando a los hogares para los que no se han hecho buenas predicciones, y analizar por un lado cómo son las características que tienen en la ola anterior, y por otro la información que se tenga sobre la ola que se intenta predecir, y comprobar si las variables que tienen más peso para explicar la participación son las de la ola corriente o las de la ola anterior.
    \item Una opción que siempre hay que considerar es un \textbf{cambio de enfoque}. Hay dos alternativas que son interesantes:
    \begin{enumerate}
        \item En la exploración de los datos vimos que los hogares que han participado sólo en una edición muestran más proporción de abandonos que los que han participado más de dos años. Una posible explicación de esto es que los hogares que han participado más de una vez están más comprometidos con el estudio, y seguramente merezca la pena \textbf{enfocar el análisis en la predicción de la participación de los paneles en su segunda ola} en vez de hacer la predicción para todos los hogares.
        \item En vez de predecir un resultado binario, de participar o no participar, se puede plantear hacer un ejercicio de \textbf{análisis de superviviencia} (survival analysis), e intentar predecir el número de ediciones en las que participará un hogar de la EFF antes de abandonar el estudio. Esta información está disponible y se podría utilizar información de todas las olas de la EFF.
    \end{enumerate}
\end{enumerate}