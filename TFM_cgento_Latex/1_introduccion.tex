\chapter{Introducción}
\label{chapter:introduccion}

Las encuestas son una forma de recolección de datos que se fundamenta en plantear preguntas relevantes a una muestra de unidades muestrales, y utilizar sus respuestas para entender a una población en su conjunto. Desde una perpectiva temporal, hay dos tipos de encuestas: las de sección cruzada, y las longitudinales o panel. Las encuestas de sección cruzada realizan sus preguntas a las unidades muestrales en un momento concreto del tiempo, mientras que las longitudinales plantean sus preguntas de manera repetida a las mismas unidades muestrales durante un período de tiempo (semanas, meses, años...). En este documento nos vamos a centrar en las encuestas longitudinales.

La dimensión temporal de las encuestas longitudinales las convierten en una herramienta muy útil para poder analizar relaciones causales, ya que recogen cambios de opiniones, comportamientos o estados de los mismos encuestados panel a lo largo del tiempo. Sin embargo, la calidad de esos análisis depende de la cooperación exitosa y continuada de dichos encuestados durante las sucesivas ediciones u olas de la encuesta. El abandono prematuro y acumulado en el tiempo de participantes en un panel se conoce como \textbf{Panel Attrition} (\cite{watson2009identifying}.

Conceptualmente, las causas del Panel Attrition pueden clasificarse en tres categorías secuenciales: no-localización, no-contacto y no-cooperación (\cite{lepkowski2002nonresponse}). La no-localización se refiere a no conseguir localizar a los panelistas durante el período de campo. Esto suele ocurrir por cambios o errores en la información de contacto recogida durante la ola anterior (direcciones de residencia, teléfonos o email). Si un participante panel cambia de residencia o cambia de número de teléfono, es más costoso o incluso inviable localizarle.

La segunda causa es el no-contacto, que es el hecho de, después de localizar exitosamente al panelista, no conseguir establecer un contacto. Para que esto suceda, es necesario que el intento de contacto por parte de los encuestadores coincida temporalmente con la disponibilidad del panelista. Esto depende completamente del método de recolección de los datos. Por ejemplo, en una entrevista personal, el panelista debe estar en su residencia justo en el mismo momento en el que el entrevistador hace la visita al hogar.

Finalmente, después de establecer el contacto, es necesario convencer al panelista para que vuelva a participar otra vez en la encuesta. La falta de cooperación en encuestas en general se ha analizado bastante en la literatura, y suelen destacar factores como las catacterísticas demográficas de los encuestados o la temática de la encuesta (para un desarrollo más detallado, puede consultarse \cite{groves1992understanding}). Sin embargo, en el caso de las encuestas longitudinales, los encuestados tienen experiencia previa por haber participado anteriormente, y este hecho puede potenciar el efecto de factores como la duración de la entrevista, la carga cognitiva por pensar las respuestas o la fatiga por haber participado en varias ediciones anteriores (\cite{laurie1999strategies}, \cite{watson2009identifying}, \cite{lynn2018tackling}).

Con respecto a las consecuencias del Panel Attriton, \cite{lynn2018tackling} destaca dos principales problemas. Por un lado, si la tasa de attrition es alta, el tamaño de la muestra se reducirá drásticamente con el paso de las olas, lo que provocará que la precisión de los estimadores de la encuesta sea muy baja, y además limitará o incluso imposibilitará el análisis de subgrupos dentro de la muestra. Por otro lado, si el Panel Attrition no es aleatorio, y por tanto los panelistas que abandonan la encuesta son sistemáticamente diferentes a los que se mantienen, existe el riesgo de introducir un sesgo de no-respuesta en los estimadores.

Tradicionalmente, los efectos del Panel Attrition se han mitigado con el uso de métodos de imputación múltiple (\cite{rubin1987multiple}), reponderando de pesos muestrales (\cite{groves2009survey}) e introduciendo muestras de refresco para sustituir a las unidades muestrales perdidas (\cite{hirano1998combining}). Pero en las últimas décadas se ha extendido el uso de los diseños adaptativos y reactivos (adaptative and responsive designs, \cite{groves2006responsive}, \cite{wagner2008adaptive}, \cite{schouten2017adaptive}, \cite{tourangeau2017adaptive}). La idea detrás de estos diseños se fundamenta en utilizar toda la información que se genera durante la elaboración de encuestas (respuestas al cuestionario, paradata u observaciones de los entrevistadores) para diseñar implementaciones informadas cuyo objetivo sea mejorar la calidad de los datos, reducir los costes, o ambos. Por ejemplo, se han utilizado para revisar incentivos a participar para grupos concretos de encuestados \cite{mcgonagle2022effects} o revisar el orden de las preguntas (\cite{early2017dynamic}. En este sentido, las encuestas longitudinales ofrecen una gran oportunidad para estos diseños porque contienen mucha información tanto de los trabajos de campo que se estén desarrollando, como los que se realizaron en olas anteriores. Por ejemplo, en \cite{kreuter2015note} utilizan los registros de llamadas a panelistas en ediciones anteriores para optimizar las estrategias de contacto en las ediciones siguientes.

Dentro de este contexto de los diseños adaptativos y reactivos, en las últimas décadas se ha desarrollado el uso algoritmos de Machine Learning en la metodología de encuestas (\cite{buskirk2018introduction}, \cite{kern2019tree}). Y de manera particular, han presentado resultados prometedores a la hora de predecir resultados de participación en los trabajos de campo, especialmente los modelos basados en árboles de decisión (\cite{kern2019tree}, \cite{kern2021predicting}, \cite{liu2020using}). En el contexto de Panel Attrition, en \cite{beste2023case} utilizaron información de ediciones pasadas de una encuesta a hogares en alemania para entrenar un Random Forest e identificar hogares panelistas con una baja propensión a participar. Luego, utilizaron esa información para crear un diseño experimental en la siguiente ola de la encuesta, en el cual se asignaba un incentivo monetario adicional a la mitad de esos hogares. Sus resultados mostraron incrementos en las tasas de respuesta de los hogares tratados, y animaron a sus responsables a seguir utilizando este diseño adaptativo en futuras ediciones.

El objetivo de este Trabajo de Fin de Máster es adaptar la implementación de machine mearning vista en \cite{beste2023case} para predecir la participación de los hogares panel en el contexto de la Encuesta Financiera de las Familias (EFF). En cada uno de estos artículos se utiliza información sobre hogares que participaron en olas pasadas de dos encuestas, y se utiliza esa información para predecir una variable binaria de participación o no participación en una ola posterior. El rendimiento de todos los modelos se compara con un modelo de referencia, que es una regresión logística. En el caso de este proyecto, utilizamos datos de las olas 4, 5, 6 y 7 de la EFF (que se corresponden con los años 2011, 2014, 2017 y 2020) para entrenar varios modelos de machine learning. A continuación, utilizamos dichos modelos para predecir la participación de los hogares panel en la ola 8 (que se corresponde con el año 2022), y comparamos su rendimiento con respecto a un modelo de regresión logistica.

El resto de este documento se organiza de la siguiente manera. En el siguiente capítulo se exponen los objetivos, la planificación y las motivaciones personales de este proyecto. En el tercer capítulo se presenta la EFF, los datos que se han utilizado y la metodología aplicada. En el cuarto capítulo se presentan los resultados de la implementación descrita en el tercer capítulo, junto con los desafíos encontrados durante el proceso. Finalmente, el último capítulo contiene las conclusiones obtenidas de los resultados del proyecto y reflexiones sobre cuáles podrían ser los próximos pasos para continuar con este proyecto en el futuro.
