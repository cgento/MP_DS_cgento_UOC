\chapter{Introducción}
\label{chapter:introduccion}

Los estudios longitudinales son proyectos de investigación en los que se hace un seguimiento a un grupo de unidades muestrales (personas, hogares...) a lo largo de un período de tiempo. En el ámbito de las ciencias sociales, la recolección de datos de muchos de estos estudios se realiza mediante el uso de encuestas, de tal manera que las mismas personas responden a las preguntas del mismo cuestionario de manera repetida durante un tiempo, que pueden ser semanas, meses o incluso años. Esto da lugar a las llamadas encuestas longitudinales o encuestas panel. La dimensión temporal de estas encuestas las convierten en una herramienta muy útil para poder analizar relaciones causales, ya que permiten observar cambios en opiniones, comportamientos o estados de los mismos panelistas a lo largo del tiempo. Sin embargo, la calidad de esos análisis depende de la cooperación exitosa y continuada de dichos panelistas durante las sucesivas ediciones u olas de la encuesta. El abandono prematuro y acumulado en el tiempo de participantes en un panel se conoce como \textbf{Panel Attrition} (\cite{watson2009identifying}).

\cite{lynn2018tackling} destaca que el Panel Attrition presenta dos principales problemas. Por un lado, si la tasa de abandonos es alta, el tamaño de la muestra se reducirá drásticamente en pocas olas, lo que provocará que la precisión de los estimadores de la encuesta sea muy baja, y además limitará o incluso imposibilitará el análisis de subgrupos dentro de la muestra. Por otro lado, si el Panel Attrition no es aleatorio y los panelistas que abandonan la encuesta son sistemáticamente diferentes a los que se mantienen, existe el riesgo de introducir un sesgo de no-respuesta en los estimadores de la encuesta.

Tradicionalmente, los métodos utilizados para mitigar los efectos del Panel Attrition se han centrado en el impacto estadístico que provoca, principalmente con el uso de métodos de imputación múltiple (\cite{rubin1987multiple}), la reponderación de pesos muestrales (\cite{groves2009survey}) e introduciendo muestras de refresco para sustituir a las unidades muestrales perdidas (\cite{hirano1998combining}). Pero en las últimas décadas ha aumentado el interés por tratarlo durante los procesos de creación y recolección de datos, lo que ha llevado a la extensión del uso de los llamados diseños adaptativos y reactivos (adaptative and responsive designs, \cite{tourangeau2017adaptive}). La idea detrás de estos diseños se fundamenta en utilizar toda la información que se genera durante la elaboración de encuestas (respuestas al cuestionario, paradata u observaciones de los entrevistadores) para diseñar implementaciones informadas cuyo objetivo sea mejorar la calidad de los datos, reducir los costes, o ambos. En este sentido, las encuestas longitudinales ofrecen una gran oportunidad para estos diseños porque contienen mucha información tanto de los trabajos de campo que se estén desarrollando, como los que se realizaron en olas anteriores. Por ejemplo, se ha utilizado información sobre intentos de contacto en olas pasadas para revisar la estrategia de incentivos para hogares a los que cuesta volver a entrevistar (\cite{mcgonagle2022effects}) o para optimizar las estrategias de contacto en las ediciones siguientes (\cite{kreuter2015note}).

Dentro de este contexto de los diseños adaptativos y reactivos, en las últimas décadas se ha desarrollado el uso algoritmos de Machine Learning en la metodología de encuestas, especialmente para predecir Panel Attrition (\cite{buskirk2018introduction}). Por ejemplo, en \cite{beste2023case} utilizan información de ediciones pasadas de una encuesta a hogares en Alemania para entrenar varios modelos basados en algoritmos de Machine Learning para identificar hogares panelistas con una baja probabilidad de volver a participar en la siguiente edición. Posteriormente, utilizan el mejor de esos modelos para predecir cuáles hogares panelistas de una nueva edición tenían menor probabilidad de colaborar de nuevo, y usan esas predicciones para crear un diseño experimental enfocado en esos hogares.

El objetivo de este Trabajo de Fin de Máster es adaptar la implementación de machine mearning vista en \cite{beste2023case} para predecir la participación de los hogares panel al caso de estudio de la Encuesta Financiera de las Familias (EFF). La EFF es una encuesta a hogares representativa de los hogares que residen en España, y es creada por el Banco de España. La idea central del ejercicio de predicción consiste en utilizar información de hogares en olas pasadas de la EFF para predecir una variable binaria de participación o no participación en la ola siguiente. Para ello, se entrenan cuatro modelos basados en algoritmos de machine learning y se compara su rendimiento con un modelo de referencia utilizado tradicionalmente en análisis de Panel Attrition, que en este caso es una Regresión Logística o Logit. En este proyecto, para el entrenamiento de los modelos se utiliza información sobre hogares que han participado en las olas 4, 5, 6 y 7 de la EFF (que se corresponden con los años 2011, 2014, 2017 y 2020), y a continuación se utilizan dichos modelos para predecir la participación de los hogares panel elegibles para la ola 8 (que se corresponde con el año 2022).

Los seis capítulos restantes de este documento se organizan de la siguiente manera. En el siguiente capítulo se exponen los objetivos, la planificación del proyecto y las motivaciones personales para realizarlo. En el tercer capítulo se hace una revisión del estado del arte para este estudio, en el que se describe cuáles son las causas del Panel Attrition, cómo se están utilizado algoritmos de machine learning para predecirlo, y finalmente se presenta la Encuesta Financiera de las Familias (EFF). En el cuarto capítulo se describe la metodología utilizada en el análisis exploratorio de datos y en el ejercicio de entrenamiento, validación y test de los modelos de predicción de Panel Attrition. En el quinto capítulo se presentan los datos de la EFF y se realizan una descripción de las etapas de producción de la EFF, y un análisis exploratorio de los datos. En el sexto capítulo se muestran y comentan los resultados del entrenamiento y la evaluación de los modelos de predicción, y se analiza la importancia de las variables utilizadas en uno de los modelos de predicción para explicar el Panel Attrition. Finalmente, en el último capítulo se exponen las conclusiones y reflexiones sobre el proyecto y cuáles pueden ser las líneas de trabajo para el futuro.