\chapter{Resultados}
\label{chapter:resultados}

\section{Análisis exploratorio de los datos}

El objetivo del análisis exploratorio de datos es investigar las características de los datos que se van a utilizar. Por un lado, se observan las características particulares de cada variable, y por otro las relaciones que existan entre ellas, y de manera especial la que tienen con la variable Attrition. Esta información es importante porque ayuda a identificar y tratar rasgos de las variables que pueden afectar a los modelos de machine learning que se quieren entrenar.

Como veremos a continuación, en este proyecto se ha manejado una gran cantidad de datos de gran diversidad de origen y formato. El análisis exploratorio de toda esa información es muy amplio y no es posible incluir todo el trabajo en esta sección. Por esa razón, sólo se muestran resultados que son de interés para el análisis del panel attrition, o resultados que han ayudado a la toma de decisiones para la selección o transformación de variables para los modelos de predicción. 

Este análisis se separa en cuatro bloques. El primero describe la gran cantidad de datos que se generan en la EFF y el proceso de filtrado que se ha realizado para este proyecto. Los otros tres boques abordan tres temas que preocupan a los productores de encuestas porque suelen afectar a la participación de los panelistas: la experiencia en la encuesta y las entrevistas, la persona que responde a la encuesta, y las características de los hogares.

\subsection*{Número de registros y variables}

Una sola edición de la EFF produce una gran cantidad de datos. El Cuadro \ref{table:registers} muestra el número de registros y variables disponibles en cada uno de los ficheros utilizados en este proyecto. Estos números incluyen a todos hogares contactados y a todas variables generadas durante la producción de datos de la EFF en sus ediciones de 2017, 2020 y 2022\footnote{De la EFF2022 sólo se utiliza el fichero de contactos ya que sólo se necesita la información sobre la participación de los hogares panel en dicha edición.}. Tras realizar el filtrado de hogares elegibles para el estudio, se obtiene que los hogares de la EFF2017 elegibles para la EFF2020 son 5,937\footnote{Originalmente se identificaron 5938 hogares de la EFF2017 elegibles para la EFF2020. Pero uno de esos hogares no tenía registros en el fichero de paradata, y se eliminó del conjunto de datos final.} y los hogares elegibles de la EFF2020 para la EFF2022 son 5,505. Con respecto al censo de entrevistadores, sus números incluyen a los 260 entrevistadores que han participado en las ediciones de 2014, 2017, 2020 y 2022. Tras filtrar por las ediciones de 2017 y 2020, se obtiene que en la EFF2017 participaron 69 entrevistadores, mientras que en la EFF2020 participaron 65 entrevistadores, de los cuales 25 personas también participaron en la EFF2017.

\begin{table}[ht]
\centering{}
\begin{tabular}{lcccc}
\cline{2-5}
                            & \multicolumn{2}{c}{\textbf{EFF2017}}        & \multicolumn{2}{c}{\textbf{EFF2020}}        \\ \cline{2-5} 
\textbf{Nombre del fichero} & \textbf{Registros}   & \textbf{Variables}   & \textbf{Registros}   & \textbf{Variables}   \\ \hline
Fichero de trabajo          & 6,413                & 6,103                & 6,313                & 6,497                \\
Fichero de datos imputados  & 6,413                & 659                  & 6,313                & 787                  \\
Fichero de contactos        & 14,456               & 640                  & 15,457               & 636                  \\
Fichero de revisión      & 44,760               & 22                   & 35,217               & 51                   \\
Fichero paradata            & 2,807,091            & 13                   & 3,121,437            & 12                   \\ \hline
                            & \multicolumn{1}{l}{} & \multicolumn{1}{l}{} & \multicolumn{1}{l}{} & \multicolumn{1}{l}{} \\
                            & \multicolumn{4}{c}{\textbf{EFF2022}}                                                      \\ \cline{2-5} 
\textbf{}                   & \multicolumn{2}{c}{\textbf{Registros}}      & \multicolumn{2}{c}{\textbf{Variables}}      \\ \cline{2-5} 
Fichero de contactos        & \multicolumn{2}{c}{15,182}                  & \multicolumn{2}{c}{636}                     \\ \hline
                            & \multicolumn{1}{l}{} & \multicolumn{1}{l}{} & \multicolumn{1}{l}{} & \multicolumn{1}{l}{} \\
\multicolumn{5}{c}{\textbf{Censo de entrevistadores}}                                                                   \\ \hline
\multicolumn{2}{c}{\textbf{Registros}}             & \multicolumn{3}{c}{\textbf{Variables}}                             \\ \hline
\multicolumn{2}{c}{260}                            & \multicolumn{3}{c}{56}                                             \\ \hline
\end{tabular}
\caption{\textit{Número de registros y variables de los ficheros de datos}}
\label{table:registers}
\end{table}

En el cuadro \ref{table:registers} también puede observarse que hay ficheros que almacenan más de 6,000 variables. Esto supone un problema de dimensionalidad ya que hay más variables que registros en los datos. Sin embargo, hay cuatro maneras para reducir drásticamente el número de variables a manejar sin perder información relevante, y obtener las variables mencionadas en el cuadro \ref{table:vars}. La primera es que la inmensa mayoría de variables almacenan las respuestas al cuestionario principal de la EFF. En el segundo párrafo de la sección \ref{section:datos} se comentó que el número de preguntas que se formulan depende del número de miembros del hogar, sus edades, y los activos y deudas que posea el hogar, y que en la EFF2017 se plantearon entre 137 y 594 preguntas a cada hogar. Como los modelos de predicción requieren de variables que contengan datos para todos los registros, es posible descartar muchas variables por no tener valores para todos los hogares.

La segunda razón para descartar variables es que muchas no son informativas en su estado original y necesitan ser combinadas con otras para poder obtener información interpretable, o se utilizan como apoyo para la imputación. Por ejemplo, la información sobre cantidades monetarias se recoge en cuatro variables que permiten declarar valores en intervalos a los hogares que no quieran o no puedan dar un valor puntual (\cite{effmethod2017}). Esto se utiliza en la imputación para estimar valores puntuales dentro del rango declarado por el hogar. Al usar el fichero de datos imputados para entrenar los modelos, todas esas variables auxiliares se descartan.

En tercer lugar, hay variables duplicadas porque están almacenadas en varios ficheros de datos, por lo que sólo es necesario extraerlas de uno de esos ficheros. Por ejemplo, todas las variables que aparecen en el fichero de datos imputados también aparecen en el fichero de trabajo. Del fichero de trabajo se extraen indicadores de no-respuesta y otras variables de interés que no aparecen en el fichero de datos imputados, y de éste último se extraen las variables con los valores missing imputados.

Finalmente, para que los modelos de predicción puedan aplicarse tanto para datos de la EFF2017 como para la EFF2020, sólo se seleccionan las variables que estaban disponibles en ambas olas. Esta tarea ha requerido una gran dedicación de esfuerzo y tiempo, ya que en algunos ficheros se detectaron variables que no mantuvieron su nomeclatura, el tipo de dato almacenado o la codificación de los datos entre diferentes olas. Para asegurar la homogeneidad, se ha revisado de manera individualizada la nomenclatura y la codificación de cada variable para ambas ediciones de la EFF.

\subsection*{La experiencia en la encuesta y las entrevistas}

En la sección \ref{section:causes_attrition} se comentó que los hogares panel poseen experiencia previa sobre la encuesta que puede afectar a su participación en olas posteriores. Esta experiencia puede abarcar varias ediciones, pero también puede ser informativo observar datos sobre la ola más reciente. En este apartado vamos a ver primero cómo fue la proporción de abandonos en 2020 para cuatro variables que dan información sobre la experiencia de los hogares en 2017, y luego un breve análisis sobre las duraciones de las entrevistas de 2017.

En la figura \ref{fig:fig1} hay cuatro gráficos de mekko que muestran cómo fue la participación en la EFF2020 de hogares elegibles de la EFF2017 según su ola de entrada en la EFF, si consintieron grabar la entrevista de la EFF2017, si dicha entrevista se realizó con un proxy, y el nivel de recelo que mostraron después de realizarla. Los gráficos de mekko son gráficos de columnas apiladas 100\% en los que la anchura de cada columna muestra la proporción de hogares que hay de una categoría dentro de la muestra. Las regiones superiores o rojas de cada columna muestran la proporción de hogares que no participaron en la EFF2020, mientras que las regiones inferiores o azules muestran la proporción de hogares que sí participaron.

\begin{figure}[h]
	\centering
	\includegraphics[width=1\textwidth]{figs/figure1.png}
	\caption{Participación en EFF2020 por experiencia en la EFF y en la entrevista de EFF2017}
	\label{fig:fig1}
\end{figure}

En la figura superior izquierda de la figura \ref{fig:fig1} se observa que la proporción de abandono de los hogares que han participado sólo en la EFF2017 es mayor que la de los que han participado en la EFF2014 y la EFF2011. De hecho, la proporción de abandonos de hogares que empezaron en esas dos ediciones es similar. Esto sugiere que los hogares que van a participar en su segunda ola podrían ser más complicados de retener que los que llevan más tiempo. Sería planteable hacer un análisis más profundo de estos hogares. La figura superior derecha muestra que la proporción de hogares que no consintieron grabar la entrevista en la EFF2017 es mayor que los que sí consintieron grabarla. Esto puede ser un signo de recelo hacia el entrevistador o los encuestadores, lo que puede dificultar la participación en las siguientes ediciones. En ese sentido también es interesante ver si los hogares se mostraban recelosos tras la entrevista, que es lo que se observa en la figura inferior derecha de la figura \ref{fig:fig1}. La gran mayoría de hogares no se mostraron recelosos tras la entrevista, pero puede observarse que la proporción de hogares que no participaron en la EFF2020 es mayor a medida que aumenta el nivel de recelo. Finalmente, en la figura inferior izquierda, se observa que la proporción de abandonos en la EFF2020 fue mayor entre los hogares que hicieron la entrevista con proxy en 2017.

\begin{figure}[h]
	\centering
	\includegraphics[width=1\textwidth]{figs/figure4.png}
	\caption{Duración por sección del cuestionario en la EFF2017}
	\label{fig:fig4}
\end{figure}

\subsection*{Características y comportamiento de PR en la EFF2017}

\begin{figure}[h]
	\centering
	\includegraphics[width=1\textwidth]{figs/figure2.png}
	\caption{Características y actitudes del PR}
	\label{fig:fig2}
\end{figure}

\subsection*{Características hogar}

\begin{figure}[h]
	\centering
	\includegraphics[width=1\textwidth]{figs/figure3.png}
	\caption{Características del hogar}
	\label{fig:fig3}
\end{figure}

En la figura 4.1 presenta cuatro gráficos de mekko que presentan las propociones de hogares panel que vuelven a participar (región azul) frente a los que no vuelven a hacerlo (región roja) para las variables de número de olas en las que se ha participado (arriba a la izquierda), el nivel de recelo después de la entrevista (arriba a la derecha), el estado de salud reportado por la PR (abajo a la izquerda) y la edad de la PR (abajo a la derecha).


En la figura puede observarse que la proporción de abandono de los hogares que han participado sólo una ola es mayor que la de los que han participado en más de una. Este resultado sugiere que los hogares que llevan menos tiempo en la encuesta podrían ser más complicados de retener, y que podría ser interesante hacer un análisis enfocado en este tipo de hogares.

El segundo resultado a destacar es el del recelo mostrado por el hogar tras la entrevista. Recordemos que en la EFF se hacen preguntan que pueden considerarse delicadas, como por ejemplo si se poseen joyas, o cuál es el saldo que el hogar tiene en su cuenta corriente. Puede ser comprensible que se genere recelo. Este se mide con tres niveles: nada receloso, algo receloso y muy receloso. En el gráfico se observa que la proporción de hogares que abandonan la EFF aumenta con el nivel de recelo. Este resultado es bastante obvio, pero es muy importante resaltarlo porque a un entrevistador podría resultarle muy útil saber si el hogar con el que está a punto de contactar se mostró muy receloso después de hacer la entrevista en la ola anterior, y en su estrategia de contacto y de ganar la cooperación podría hacer más énfasis en los aspectos de la encuesta que causan más recelo.

Finalmente, los dos gráficos de la parte de inferior hacen referencia a características de la PR. En el de la izquierda se observa que la proporción de hogares que abandona la encuesta es mayor cuando menor es el nivel de salud que reporta la PR. De nuevo, esto puede parecer algo obvio porque alguien que tiene peor salud seguramenten o quiera gastar su tiempo en realizar una encuesta. Pero, de nuevo, la estrategia de contacto y de ganar la cooperación de un entrevistador puede adaptarse si ya saben que van a ver a alguien que seguramente tenga mala salud, ya sea ofreciendo hacer la entrevista en varias sesiones, o en un ambiente en el que la PR se muestre más cómoda.

Finalmente, el último gráfico muestra el nivel de interés que mostró la PR durante la entrevista durante la edición anterior. Se observa que la proporción de hogares que vuelven a participar es mayor cuanto mayor es el interés que mostraron durante la ola anterior. De nuevo, se trata de un resultado obvio. Pero, al igual que con el nivel de recelo, este podría ser un dato que podría ser muy útil para el entrevistador cuando esté preparando su estategia de contacto con el hogar.

\section{Evaluación de modelos}

El cuadro 4.3 contiene los resultados del rendimiento de los modelos entrenados sobre el conjunto de test, es decir, los resultados para predecir la no participación de los panelistas en la EFF2022. Las métricas de evaluación que se utilizan son Accuracy, Precision, Recall, F1 y ROC AUC. La métrica de referencia que utilizamos para la evaluación es la ROC AUC, que se encuentra en la parte derecha del cuadro. Esta métrica mide el rendimiento entre la tasa de falsos positivos y falsos negativos. Toma valores de 0.5 a 1, con 1 siendo un predictor perfecto, y 0.5 el que se obtendría con una estimación realizada de manera aleatoria.

\begin{table}[ht]
    \centering
    \begin{tabular}{lccccc}
    \hline
        \textbf{Modelo} & \textbf{Accuracy} & \textbf{Precision} & \textbf{Recall} & \textbf{F1} & \textbf{ROC AUC} \\ \hline
        Logit & 0,6556 & 0,3749 & 0,3573 & 0,3659 & 0,5959 \\ 
        CART & 0,6434 & 0,3338 & 0,2835 & 0,3066 & 0,5521 \\ 
        Random Forest & 0,6489 & 0,3529 & 0,3148 & 0,3328 & 0,5821 \\ 
        XGBooster & 0,6718 & 0,3772 & 0,2769 & 0,3194 & 0,5911 \\ 
        Naive Bayes & 0,6254 & 0,3504 & 0,4063 & 0,3763 & 0,5798 \\ \hline
    \end{tabular}
    \caption{Métricas de evaluación de los modelos de predicción en el conjunto de test}
\end{table}


El modelo de referencia de este estudio, el Logit, presenta una ROC AUC de 0,5959. Se considera que un valor inferior a 0,6 es un resultado malo, por lo que el modelo Logit no es un buen predictor. Con respecto a los otros modelos, observamos que el CART y el Naïve Bayes presentan valores más bajos que el Logit, de 0,5521 y 0,5798 respectivamente. El Random Forest y el XGBooster, en cambio, presentan valores de 0,5821 y 0,5911 respectivalente, que son ligeramente superiores al del modelo Logit.

De estos resultados podemos ver que los modelos de Random Forest y XGBooster mejoran al Logit de referencia. Y entre estos dos, el Random Forest presenta valores un poco más altos en Accuracy y en Precision, y el XGBooster es mejor en Recall y F1. Aun así, es necesario recalcar que el valor de ROC AUC sigue estando por debajo de 0,6, por lo que, aunque se mejore el rendimiento con respecto al modelo Logit, los resultados de los test son malos.

Estos resiltados indican que hay modelos de machine learning que mejoran en rendimiento de la predicción con respecto a un Logit. Sin embargo, las métricas señalan que el rendimiento del mejor modelo es regular. La pregunta que hay que preguntarse entonces es, ¿por qué no está funcionando bien la predicción?

Una de las causas puede ser el overfitting, es decir, que los modelos no son capaces de generalizar bien porque se han adaptado demasiado bien a los datos con los que han sido entrenados, y por tanto su rendimiento no es bueno cuando se les presenta con datos nuevos. una manera de comprobar esto es haciendo el ejercicio predicción sobre los datos de entrenamiento con estos mismos modelos, y ver cómo son esos resultados. Estos resultados pueden verse en el cuadro 4.4.

\begin{table}[ht]
    \centering
    \begin{tabular}{lccccc}
    \hline
        \textbf{Modelo} & \textbf{Accuracy} & \textbf{Precision} & \textbf{Recall} & \textbf{F1} & \textbf{ROC AUC} \\ \hline
        Logit & 0,6389 & 0,4905 & 0,4518 & 0,4704 & 0,6517 \\ 
        CART & 0,6126 & 0,4594 & 0,5178 & 0,4868 & 0,6188 \\ 
        Random Forest & 0,6596 & 0,5223 & 0,4770 & 0,4986 & 0,6726 \\ 
        XGBooster & 0,6544 & 0,5144 & 0,4675 & 0,4898 & 0,6722 \\ 
        Naive Bayes & 0,6251 & 0,4741 & 0,5178 & 0,4950 & 0,6435 \\ \hline
    \end{tabular}
    \caption{Métricas de evaluación de los modelos de predicción en el conjunto de entrenamiento}
\end{table}

Como era de esperar, los resultados de las predicciones de los modelos sobre los datos de entrenamiento son mejores que los que se observan con respecto a los datos test. En este caso, el modelo que presenta mejor rendimiento es el Random Forest, con una métrica de ROC AUC de 0,6726, ligeramente superior a la del XGBooster. El resto de métricas del Ranfom Forest también son ligeramente mejores que las del XGBooster. Sin embargo, esta métrica de ROC AUC está entre 0,6 y 0,75, que lo clasifica como un test regular.

Continuando con el Random Forest, una ventaja de los modelos basados en árboles con respecto a otros modelos es que pueden ser interpretados. En el caso del Radom Forest, es posible consultar qué variables han tenido más peso a la hora de clasificar la participación de los hogares. Aunque los resultados de la predicción del training no sean buenos, merece la pena echarle un ojo por los patrones que haya podido detectar. El cuadro 4.5 presenta las 10 variables con más importancia en el modelo de Random Forest.

\begin{table}[ht]
    \centering
    \begin{tabular}{lc}
    \hline
        \textbf{Variable} & \textbf{Importancia} \\ \hline
        PR es panel & 0,0822 \\ 
        Interés PR & 0,0790 \\ 
        Razones para colaborar - Relevancia de la encuesta & 0,0605 \\ 
        Ratio amplio & 0,0594 \\ 
        Poseen vehículos & 0,0539 \\ 
        Tienen deuda & 0,0496 \\ 
        Situación laboral - Asalariado & 0,0475 \\ 
        Razones colaborar - Interesado en estos estudios & 0,0380 \\ 
        Posee planes de pensiones & 0,0353 \\ 
        PR consiente grabar la entrevista & 0,0348 \\ \hline
    \end{tabular}
    \caption{Selección de variables para entrenar los modelos de predicción}
\end{table}

La importancia de una variable en el Random Forest mide el peso relativo que ha tenido una variable concreta a la hora de crear las ramificaciones de los diferentes árboles de decisiones que va generando el Random Forest durante su entrenamiento. En este caso de la participación en la EFF2020, las tres variables que más importancia tuvieron fueron que la PR fuera panel, que la PR mostrase interés durante la entrevista, y que el entrevitador indicase que el hogar colaboró por la relevancia de la encuesta. También es interesante destacar la variable ratio amplio, que es un indicador de la no-respuesta del hogar a la hora de facilitar cantidades monerarias\footnote{El ratio amplio se define como el cociente entre el número de preguntas en euros respondidas por el hogar, ya fuera como valor puntual o como intervalos, sobre el número total de preguntas planteadas. Cuando mayor es su valor, más información ha facilitado el hogar.}. Finalmente, otra variable que es importante para hacer la clasificación es si la PR consintió que se grabase la entrevista.