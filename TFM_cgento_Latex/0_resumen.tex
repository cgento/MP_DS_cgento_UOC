\pagenumbering{roman} 
\setcounter{page}{1} 
\pagestyle{plain}

%%%%%%%%%%%%%%%%
%%% CREDITOS %%%
%%%%%%%%%%%%%%%%
\chapter*{Créditos/Copyright}

\vspace{1cm}

\begin{figure}[ht]
    \centering
	\includegraphics[scale=1]{images/license.png}
\end{figure}

Esta obra está sujeta a una licencia de Reconocimiento -  NoComercial - SinObraDerivada

\href{https://creativecommons.org/licenses/by-nc-nd/3.0/es/}{3.0 España de CreativeCommons}.

%%%%%%%%%%%%%
%%% FICHA %%%
%%%%%%%%%%%%%
\chapter*{FICHA DEL TRABAJO FINAL}

\begin{table}[ht]
	\centering{}
	\renewcommand{\arraystretch}{2}
	\begin{tabular}{r | p{10cm}}
		\hline
		Título del trabajo: & Predecir panel attrition con Machine Learning: Un análisis con la Encuesta Financiera de las Familias\\
		\hline
        Nombre del autor: & Carlos Luis Gento de Celis\\
		\hline
        Nombre del colaborador/a docente: & Jordi Escayola Mansilla\\
		\hline
        Nombre del PRA: & Antonio Lozano Bagén\\
		\hline
        Fecha de entrega (mm/aaaa): & 10/2023\\
		\hline
        Titulación o programa: & Máster Universitario en Ciencia de Datos\\
		\hline
        Área del Trabajo Final: & Informática, Multimedia y Telecomunicación\\
		\hline
        Idioma del trabajo: & Español\\
		\hline
        Palabras clave & predictive models, machine learning, panel attrition\\
		\hline
	\end{tabular}
\end{table}

%%%%%%%%%%%%%%%%
%%% RESUMEN  %%%
%%%%%%%%%%%%%%%%

\chapter*{Resumen}
\addcontentsline{toc}{chapter}{Resumen}

\onehalfspacing

El abandono de panelistas en encuestas longitudinales, también conocido como Panel Attrition, es un asunto preocupante porque puede sesgar y afectar la eficiencia de los resultados de estas encuestas. En este contexto, en las últimas décadas ha aumentado el uso de diseños adaptativos y reactivos. Estos diseños utilizan información de panelistas recogida en olas anteriores para desarrollar intervenciones en los procesos de creación de datos con el objetivo de disminuir el Panel Attrition en olas posteriores. Los modelos de predicción basados en algoritmos de machine learning son herramientas interesantes para el diseño de estas implementaciones porque han mostrado buenos resultados para predecir Panel Attrition en encuestas.

Tomando como referencia la implementación hecha en \cite{beste2023case}, este Trabajo de Fin de Máster busca desarrollar un modelo basado en algoritmos de machine learning para predecir Panel Attrition en el caso de la la Encuesta Financiera de las Familias (EFF) utilizando información de hogares panel de olas anteriores. Este proyecto se divide en tres partes. Primero, se realiza un análisis exploratorio de los datos generados durante la producción de la EFF, con especial interés en la relación entre las variables y el Panel Attrition. En la segunda parte, se entrenan cuatro modelos basados en algoritmos de machine learning con información de los hogares de la ola anterior para predecir Panel Attrition en la ola siguiente, se evalúan utilizando datos nuevos, y se compara su rendimiento con el de un modelo de Regresión Logística o Logit. Finalmente, se analizan todos los resultados y se consideran posibles líneas de trabajo para el futuro.

\vspace{1.5cm}

\textbf{Palabras clave}: predictive models, panel attrition, machine learning, household surveys

\chapter*{Abstract}
\addcontentsline{toc}{chapter}{Abstract}

\onehalfspacing

The drop-out of panel members in longitudinal surveys, also known as Panel Attrition, is a worrying issue because it potentially bias survey estimates and affects their efficiency. In this context, during the last decades the use of adaptative and responsive designs has increased. These designs use panelist's information from previous waves to develop informed interventions in data production processes aimed at decreasing Panel Attrition in subsequent waves. Prediction models based on machine learning algorithms are interesting tools for designing these implementations because they have shown good performance at predicting Panel Attrition.

Inspired by the implementation at \cite{beste2023case}, this Master Project aims to develop a model based on machine learning algorithms for predicting Panel Attrition in the case of the Survey os Spanish Household Finances (EFF) using panel household information from previous waves. This project is divided in three parts. Firstly, an exploration analysis is performed on all the data generated during the production of the EFF, with special focus on the relationship between each variable and Panel Attrition. Secondly, four machine-learning-based models are trained to predict Panel Attrition using household's information from the previous wave, then they are tested using new data, and their performance is compared with the one shown by a Logistic Regression or Logit. Finally, all results are analized and future lines of work are discussed.

\vspace{1.5cm}

\textbf{Key words}: predictive models, panel attrition, machine learning, longitudinal surveys, household surveys