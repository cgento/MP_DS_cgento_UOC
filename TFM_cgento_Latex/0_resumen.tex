\pagenumbering{roman} 
\setcounter{page}{1} 
\pagestyle{plain}

%%%%%%%%%%%%%%%%
%%% CREDITOS %%%
%%%%%%%%%%%%%%%%
\chapter*{Créditos/Copyright}

\vspace{1cm}

\begin{figure}[ht]
    \centering
	\includegraphics[scale=1]{images/license.png}
\end{figure}

Esta obra está sujeta a una licencia de Reconocimiento -  NoComercial - SinObraDerivada

\href{https://creativecommons.org/licenses/by-nc-nd/3.0/es/}{3.0 España de CreativeCommons}.

%%%%%%%%%%%%%
%%% FICHA %%%
%%%%%%%%%%%%%
\chapter*{FICHA DEL TRABAJO FINAL}

\begin{table}[ht]
	\centering{}
	\renewcommand{\arraystretch}{2}
	\begin{tabular}{r | p{10cm}}
		\hline
		Título del trabajo: & Predecir panel attrition con Machine Learning: Un análisis con la Encuesta Financiera de las Familias\\
		\hline
        Nombre del autor: & Carlos Luis Gento de Celis\\
		\hline
        Nombre del colaborador/a docente: & Jordi Escayola Mansilla\\
		\hline
        Nombre del PRA: & Antonio Lozano Bagén\\
		\hline
        Fecha de entrega (mm/aaaa): & 10/2023\\
		\hline
        Titulación o programa: & Máster Universitario en Ciencia de Datos\\
		\hline
        Área del Trabajo Final: & Informática, Multimedia y Telecomunicación\\
		\hline
        Idioma del trabajo: & Español\\
		\hline
        Palabras clave & predictive models, machine learning, panel attrition\\
		\hline
	\end{tabular}
\end{table}

%%%%%%%%%%%%%%%%
%%% RESUMEN  %%%
%%%%%%%%%%%%%%%%

\chapter*{Resumen}
\addcontentsline{toc}{chapter}{Resumen}

\onehalfspacing

El abandono de hogares panel en encuestas longitudinales, también conocido como Panel Attrition, es un asunto preocupante porque puede sesgar e incluso invalidar los resultados de las encuestas. En este contexto, se ha extendido el uso de información de panelistas en ediciones anteriores de encuestas para diseñar implementaciones enfocadas a aumentar la participación de esos mismos panelistas en ediciones posteriores. En los últimos últimos 


La Encuesta Financiera de las Familias (EFF) es una encuesta bienal cuyo objetivo es recoger información sobre la situación económico-financiera de los hogares que residen en España, y su evolución a lo largo del tiempo. Para ello, los hogares seleccionados pueden participar en hasta cuatro olas consecutivas de la encuesta. Sin embargo, hay hogares que abandonan el estudio antes de tiempo. Aunque estos abandonos no interrumpan el estudio, sí podrían afectar a sus resultados si el número de abandonos es demasiado grande, o si se concentra en colectivos específicos de la población. Es importante analizar las causas de estos abandonos y desarrollar herramientas que puedan evitarlos.

Este documento, en primer lugar, analiza las características de los hogares que participaron en la Encuesta Financiera de las Familias (EFF) en sus ediciones de 2017 y 2020, y si participaron en las olas de 2020 y de 2022. A continuación, plantea una serie modelos de predicción basados en métodos de Machine Learning y evalúa su capacidad para predecir si un hogar abandonará la EFF en 2020 o en 2022. Finalmente, interpreta los resultados el modelo que mejor ha funcionado.

\vspace{1.5cm}

\textbf{Palabras clave}: predictive models, panel attrition, machine learning, household surveys

\chapter*{Abstract}
\addcontentsline{toc}{chapter}{Abstract}

\onehalfspacing

The Spanish Survey of Household Finances (EFF) is a biennial survey whose goal is to collect information about the economical and financial situation of households in Spain, and its evolution through time. To do so, selected households may take part in up to four consecutive waves of this survey. However, some households cease their participation prematurely. Although withdrawals do not interrupt the research study, they might affect its results if the number of withdrawals is too high, or if it is more likely to happen for certain groups of people. It is important to analyse the causes of this phenomenon and develop tools to prevent it.

Firstly, this paper analyses the characteristics of households that responded to the EFF in its editions of 2017 and 2020, and their participation during the 2020 or 2022 waves. Next, a series of predictive models based on machine learning methods are considered and evaluated for the exercise of predicting panel attrition in the EFF. Finally, it interprets the results of the most successful model.

\vspace{1.5cm}

\textbf{Key words}: predictive models, panel attrition, machine learning, longitudinal surveys, household surveys